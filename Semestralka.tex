% Options for packages loaded elsewhere
\PassOptionsToPackage{unicode}{hyperref}
\PassOptionsToPackage{hyphens}{url}
%

\documentclass[
12pt,
a4paper,
pdftex,
czech,
titlepage
]{report}

\usepackage{float}
\usepackage{caption}
\usepackage{graphicx}
\usepackage{amsmath,amssymb}
\usepackage{lmodern}
\usepackage{iftex}
\usepackage[czech]{babel}
\ifPDFTeX
  \usepackage[T1]{fontenc}
  \usepackage[utf8]{inputenc}
  \usepackage{textcomp} % provide euro and other symbols
\else % if luatex or xetex
  \usepackage{unicode-math}
  \defaultfontfeatures{Scale=MatchLowercase}
  \defaultfontfeatures[\rmfamily]{Ligatures=TeX,Scale=1}
\fi
% Use upquote if available, for straight quotes in verbatim environments
\IfFileExists{upquote.sty}{\usepackage{upquote}}{}
\IfFileExists{microtype.sty}{% use microtype if available
  \usepackage[]{microtype}
  \UseMicrotypeSet[protrusion]{basicmath} % disable protrusion for tt fonts
}{}
\makeatletter
\@ifundefined{KOMAClassName}{% if non-KOMA class
  \IfFileExists{parskip.sty}{%
    \usepackage{parskip}
  }{% else
    \setlength{\parindent}{0pt}
    \setlength{\parskip}{6pt plus 2pt minus 1pt}}
}{% if KOMA class
  \KOMAoptions{parskip=half}}
\makeatother
\usepackage{xcolor}
\IfFileExists{xurl.sty}{\usepackage{xurl}}{} % add URL line breaks if available
\IfFileExists{bookmark.sty}{\usepackage{bookmark}}{\usepackage{hyperref}}
\hypersetup{
  hidelinks,
  pdfcreator={LaTeX via pandoc}}
\urlstyle{same} % disable monospaced font for URLs
\usepackage{color}
\usepackage{fancyvrb}
\newcommand{\VerbBar}{|}
\newcommand{\VERB}{\Verb[commandchars=\\\{\}]}
\DefineVerbatimEnvironment{Highlighting}{Verbatim}{commandchars=\\\{\}}
% Add ',fontsize=\small' for more characters per line
\newenvironment{Shaded}{}{}
\newcommand{\AlertTok}[1]{\textcolor[rgb]{1.00,0.00,0.00}{\textbf{#1}}}
\newcommand{\AnnotationTok}[1]{\textcolor[rgb]{0.38,0.63,0.69}{\textbf{\textit{#1}}}}
\newcommand{\AttributeTok}[1]{\textcolor[rgb]{0.49,0.56,0.16}{#1}}
\newcommand{\BaseNTok}[1]{\textcolor[rgb]{0.25,0.63,0.44}{#1}}
\newcommand{\BuiltInTok}[1]{#1}
\newcommand{\CharTok}[1]{\textcolor[rgb]{0.25,0.44,0.63}{#1}}
\newcommand{\CommentTok}[1]{\textcolor[rgb]{0.38,0.63,0.69}{\textit{#1}}}
\newcommand{\CommentVarTok}[1]{\textcolor[rgb]{0.38,0.63,0.69}{\textbf{\textit{#1}}}}
\newcommand{\ConstantTok}[1]{\textcolor[rgb]{0.53,0.00,0.00}{#1}}
\newcommand{\ControlFlowTok}[1]{\textcolor[rgb]{0.00,0.44,0.13}{\textbf{#1}}}
\newcommand{\DataTypeTok}[1]{\textcolor[rgb]{0.56,0.13,0.00}{#1}}
\newcommand{\DecValTok}[1]{\textcolor[rgb]{0.25,0.63,0.44}{#1}}
\newcommand{\DocumentationTok}[1]{\textcolor[rgb]{0.73,0.13,0.13}{\textit{#1}}}
\newcommand{\ErrorTok}[1]{\textcolor[rgb]{1.00,0.00,0.00}{\textbf{#1}}}
\newcommand{\ExtensionTok}[1]{#1}
\newcommand{\FloatTok}[1]{\textcolor[rgb]{0.25,0.63,0.44}{#1}}
\newcommand{\FunctionTok}[1]{\textcolor[rgb]{0.02,0.16,0.49}{#1}}
\newcommand{\ImportTok}[1]{#1}
\newcommand{\InformationTok}[1]{\textcolor[rgb]{0.38,0.63,0.69}{\textbf{\textit{#1}}}}
\newcommand{\KeywordTok}[1]{\textcolor[rgb]{0.00,0.44,0.13}{\textbf{#1}}}
\newcommand{\NormalTok}[1]{#1}
\newcommand{\OperatorTok}[1]{\textcolor[rgb]{0.40,0.40,0.40}{#1}}
\newcommand{\OtherTok}[1]{\textcolor[rgb]{0.00,0.44,0.13}{#1}}
\newcommand{\PreprocessorTok}[1]{\textcolor[rgb]{0.74,0.48,0.00}{#1}}
\newcommand{\RegionMarkerTok}[1]{#1}
\newcommand{\SpecialCharTok}[1]{\textcolor[rgb]{0.25,0.44,0.63}{#1}}
\newcommand{\SpecialStringTok}[1]{\textcolor[rgb]{0.73,0.40,0.53}{#1}}
\newcommand{\StringTok}[1]{\textcolor[rgb]{0.25,0.44,0.63}{#1}}
\newcommand{\VariableTok}[1]{\textcolor[rgb]{0.10,0.09,0.49}{#1}}
\newcommand{\VerbatimStringTok}[1]{\textcolor[rgb]{0.25,0.44,0.63}{#1}}
\newcommand{\WarningTok}[1]{\textcolor[rgb]{0.38,0.63,0.69}{\textbf{\textit{#1}}}}
\usepackage{longtable,booktabs,array}
\usepackage{calc} % for calculating minipage widths
% Correct order of tables after \paragraph or \subparagraph
\usepackage{etoolbox}
\makeatletter
\patchcmd\longtable{\par}{\if@noskipsec\mbox{}\fi\par}{}{}
\makeatother
% Allow footnotes in longtable head/foot
\IfFileExists{footnotehyper.sty}{\usepackage{footnotehyper}}{\usepackage{footnote}}
\makesavenoteenv{longtable}
\usepackage{graphicx}
\makeatletter
\def\maxwidth{\ifdim\Gin@nat@width>\linewidth\linewidth\else\Gin@nat@width\fi}
\def\maxheight{\ifdim\Gin@nat@height>\textheight\textheight\else\Gin@nat@height\fi}
\makeatother
% Scale images if necessary, so that they will not overflow the page
% margins by default, and it is still possible to overwrite the defaults
% using explicit options in \includegraphics[width, height, ...]{}
%\setkeys{Gin}{width=\maxwidth,height=\maxheight,keepaspectratio}
% Set default figure placement to htbp
\makeatletter
\def\fps@figure{htbp}
\makeatother
\setlength{\emergencystretch}{3em} % prevent overfull lines
\providecommand{\tightlist}{%
  \setlength{\itemsep}{0pt}\setlength{\parskip}{0pt}}
\setcounter{secnumdepth}{-\maxdimen} % remove section numbering
\ifLuaTeX
  \usepackage{selnolig}  % disable illegal ligatures
\fi

\usepackage{etoolbox}
\makeatletter
\patchcmd{\chapter}{\if@openright\cleardoublepage\else\clearpage\fi}{}{}{}
\makeatother

\usepackage[czech]{babel}
\usepackage[utf8]{inputenc}
\usepackage{lmodern}
\usepackage{textcomp}
\usepackage[T1]{fontenc}
\usepackage{amsfonts}
\usepackage{titlesec}
\usepackage{graphicx}

\author{Jiří Velek}
\date{30. 10. 2021}

\titleformat{\chapter}
  {\normalfont\LARGE\bfseries}{\thechapter}{1em}{}
\titlespacing*{\chapter}{0pt}{0ex plus 1ex minus .2ex}{2.0ex plus .2ex}

\begin{document}

\begin{titlepage}
	\vspace*{-2cm}
	{\centering\includegraphics[scale=2.0]{logo.pdf}\par}
	\centering
	\vspace*{2cm}
	{\Large Semestrální práce z KIV/PRO\par}
	\vspace{1.5cm}
	{\Huge\bfseries Hledání všech k nejbližších sousedů\par}
	\vspace{2cm}

	{\Large Jiří Velek\par}
	{\Large A20B0269P\par}
	{\Large jvelek@students.zcu.cz\par}

	\vfill

	{\Large 30.\,10.\,2021}
\end{titlepage}

\hypertarget{zaduxe1nuxed}{%
\chapter{Zadání}\label{zaduxe1nuxed}}

Najděte v anglicky psané odborné literatuře článek v délce alespoň 5
stránek o nějakém
algoritmu řešícím libovolný problém. Algoritmus popište do českého
referátu tak, aby ho
podle vašeho názoru pochopil běžný student 2. ročníku informatiky. Váš
text musí svědčit o tom, že algoritmu rozumíte, a musí ho z něj pochopit
i nezasvěcený čtenář.

\hypertarget{uxfavod}{%
\chapter{Úvod}\label{uxfavod}}

Existuje mnoho algoritmů pro hledání \(k\) nejbližších sousedů, plno z
nich využívají techniky jako je rozděl a panuj, nebo Voronoi diagramy \cite{edelsbrunner, rourke, preparatafp}.
Náš algoritmus pracuje bez složitého předzpracování.

\hypertarget{popis-a-definice-probluxe9mu}{%
\chapter{Popis a definice problému}\label{popis-a-definice-probluxe9mu}}

Problém je nalezení algoritmu, který k množině \(n\) bodů
\(P[x_i, y_i]\), \(i = 1,\ldots{,n}\) a množině indexů
\(j_1,\ldots{,m}\), \(1 \leq{j_k} \leq{n}\) najde k množině
\(P_{j_{1}},\ldots,{P_{j_{m}}}\) \(k\) nejbližších sousedů.

\hypertarget{popis-algoritmu}{%
\chapter{Popis algoritmu}\label{popis-algoritmu}}

Nejprve je potřeba si připravit uniformní mřížku, do které jsou vloženy
všechny body. Mřížka je navržena tak, aby se do jedné buňky vešlo více
bodů. Požadavky na mřížku jsou následující:

\begin{itemize}
\tightlist
\item
  mřížku lze vytvořit v lineárním čase
\item
  výkon se výrazně neliší v závislosti na tom, jestli jsou data
  uniformně rozložena nebo ne
\item
  velikost mřížky se dá jednoduše upravit pro jakákoliv praktická data
\end{itemize}

Všechny body jsou vloženy do mřížky tak, aby byly zachovány vzdálenosti
mezi jednotlivými body.

Vyhledávání sousedů probíhá po vrstvách, začne se od bodu \(P_{j_{r}}\)
a postupuje se ``v kružnici'' od daného bodu. Vyhledávání skončí, když
\(k\)-tý nejbližší bod má menší vzdálenost, než vzdálenost kandidáta
\(P\) a nejbližší stěny buňky, ve které \(P\) leží plus poloměr
kružnice, ve které se momentálně algoritmus nachází.

Při vyhledávání se vzdálenosti bodů uchovávají v seřazeném poli, když algoritmus najde bod s kratší vzdáleností, než
poslední prvek v tomto poli, nahradí ho nalezenou vzdáleností a znovu pole seřadí.

Po dokončení hlavní smyčky algoritmu obsahuje pole vzdáleností \(k\)
nejbližších sousedů. Indexy bodů ke kterým tyto vzdálenosti patří jsou
daní sousedé.

\hypertarget{sestrojeni_mrizky}{%
\chapter{Sestrojení uniformní mřížky}\label{sestrojeni_mrizky}}

Mřížku je třeba sestrojit tak, aby její jednotlivé buňky neobsahovali
příliš mnoho bodů. Algoritmus najde maximální a minimální x a y souřadnice
bodu (\(x_{max}, x_{min}\)) a \((y_{max}, y_{min})\)

Jedna buňka má potom velikost:
\[velikost = \alpha \sqrt{\frac{(x_{max} - x_{min})(y_{max} - y_{min})}{n}}\], kde
parametr \(\alpha\) umožňuje upravit měřítko mřížky. Šířku a výšku
mřížky lze spočítat jako:
\[xres = \Bigl\lfloor{\frac{x_{max}-x_{min}}{velikost}}\Bigr\rfloor\]
\[yres = \Bigl\lfloor{\frac{y_{max}-y_{min}}{velikost}}\Bigr\rfloor\]

Pro vložení bodů do mřížky je potřeba vypočítat pro každý bod
\(P[x, y]\) jeho souřadnice buňky \(i, j\)
\[i = \Bigl\lfloor{\frac{x - x_{min}}{xres}} \Bigr\rfloor\]
\[j = \Bigl\lfloor{\frac{y - y_{min}}{yres}} \Bigr\rfloor\]

\hypertarget{pseudokuxf3d}{%
\chapter{Pseudokód}\label{pseudokuxf3d}}

vstup: množina bodů P, množina indexů j

výstup: dvourozměrné pole, \(c[m][k]\) s \(k\) nejbližšími sousedy

\begin{Shaded}
\begin{Highlighting}[]
\ControlFlowTok{for}\NormalTok{ r }\OperatorTok{=} \DecValTok{1}\NormalTok{ to m by }\DecValTok{1}\NormalTok{:}
\NormalTok{  (ic, jc) }\OperatorTok{=}\NormalTok{ indexy v mřížce bodu Pjr}
\NormalTok{  dsh }\OperatorTok{=}\NormalTok{ vzdálenost od bodu Pjr k nejbližší stěne buňky (ic, jc)}
    
  \ControlFlowTok{for}\NormalTok{ i }\OperatorTok{=} \DecValTok{1}\NormalTok{ to k by }\DecValTok{1}\NormalTok{:}
\NormalTok{    bucket[i] }\OperatorTok{=}\NormalTok{ infinity}
\NormalTok{  dmin }\OperatorTok{=}\NormalTok{ infinity}
\NormalTok{  l }\OperatorTok{=} \DecValTok{0}
        
  \ControlFlowTok{while}\NormalTok{ dmin }\OperatorTok{\textgreater{}}\NormalTok{ dsh:}
\NormalTok{    il }\OperatorTok{=} \BuiltInTok{max}\NormalTok{(}\DecValTok{1}\NormalTok{, ic }\OperatorTok{{-}} \NormalTok{l}\NormalTok{)}
\NormalTok{    jl }\OperatorTok{=} \BuiltInTok{max}\NormalTok{(}\DecValTok{1}\NormalTok{, jc }\OperatorTok{{-}}\NormalTok{ l)}
\NormalTok{    ih }\OperatorTok{=} \BuiltInTok{min}\NormalTok{(xres, ic }\OperatorTok{+} \NormalTok{l}\NormalTok{)}
\NormalTok{    jh }\OperatorTok{=} \BuiltInTok{min}\NormalTok{(yres, jc }\OperatorTok{+}\NormalTok{ l)}
            
    \ControlFlowTok{for}\NormalTok{ i }\OperatorTok{=}\NormalTok{ il to ih by }\DecValTok{1}\NormalTok{:}
      \ControlFlowTok{if}\NormalTok{ i }\OperatorTok{==}\NormalTok{ il }\KeywordTok{or}\NormalTok{ i }\OperatorTok{==}\NormalTok{ ih:}
\NormalTok{        ji }\OperatorTok{=} \DecValTok{1}
      \ControlFlowTok{else}\NormalTok{:}
\NormalTok{        ji }\OperatorTok{=}\NormalTok{ jh }\OperatorTok{{-}}\NormalTok{ jl}
                
      \ControlFlowTok{for}\NormalTok{ j }\OperatorTok{=}\NormalTok{ jl to jh by ji:}
        \ControlFlowTok{if}\NormalTok{ cell[i][j] ještě nebyla navštívena a není prázdná:         }
\NormalTok{          dis }\OperatorTok{=}\NormalTok{ vzdálenost bodu Pjr a bodů v cell[i][j]}
            
          \ControlFlowTok{if}\NormalTok{ dis }\OperatorTok{\textless{}}\NormalTok{ bucket[k]:}
\NormalTok{            bucket[k] }\OperatorTok{=}\NormalTok{ dis}
\NormalTok{            ind[k] }\OperatorTok{=}\NormalTok{ index nalezeného bodu}
\NormalTok{            seřadit bucket a ind}
              
\NormalTok{          označit cell[i][j] jako navštívena}
            
  \NormalTok{  dmin }\OperatorTok{=}\NormalTok{ bucket[k]}
  \NormalTok{  dsh }\OperatorTok{=}\NormalTok{ dsh }\OperatorTok{+}\NormalTok{ velikost}
  \NormalTok{  l }\OperatorTok{=}\NormalTok{ l }\OperatorTok{+} \DecValTok{1}
        
  \ControlFlowTok{for}\NormalTok{ s }\OperatorTok{=} \DecValTok{1}\NormalTok{ to k by }\DecValTok{1}\NormalTok{:}
\NormalTok{    c[r][s] }\OperatorTok{=}\NormalTok{ ind[s]}
\end{Highlighting}
\end{Shaded}

\hypertarget{sloux17eitost-algoritmu}{%
\chapter{Složitost algoritmu}\label{sloux17eitost-algoritmu}}

Algoritmus byl otestován na několika různých datech. Data set č. 1
obsahuje nejjednodušší případy, kdy jsou body uniformně rozděleny. Data
set č. 2 obsahuje velké mezery mezi body, a díru ve středu. Data set č.
3a obsahuje případ, kdy se sousedé hledají pro izolovaný bod. Data set
č. 3b jsou stejná jako data set 3a s jedinou vyjímkou - sousedé se
vyhledávají pro bod mimo díru. Data set č. 4 obsahuje všechny body ve
dvou ``množinách'' a vyhledávání probíhá v jedné z nich. V data setech
na obrázcích bylo použito \(n = 1000\) a \(k = 50\)

Tabulka 1 popisuje zaplnění uniformní mřížky

\begin{figure}[htp]
\centering
\begin{longtable}[]{@{}ccccc@{}}
\toprule
Data set č. & 1 & 2 & 3(a, b) & 4 \\
\midrule
\endhead
využitých buněk & 62.8\% & 39.6\% & 51.6\% & 28.9\% \\
\bottomrule
\end{longtable}
\captionsetup{labelformat=empty}
\caption{Tabulka 1: Zaplnění buněk v mřížce \cite{base}}
\end{figure}

Pro data č. 1 a 2 má algoritmus lineární složitost, i přesto, že mřížka
v druhém případě obsahuje \(60.4\%\) nevyužitých buněk.

Pokud je bod, pro který jsou sousedé hledány, umístěn v izolaci (jako v
případě č. 3a), časová složitost algoritmu se zhorší. Při vybrání jiného
bodu (případ 3b) je složitost opět lineární.

Případ č. 4 nedělá žádné potíže (složitost je lineární), i přesto, že
\(71.1\%\) buněk je prázdných.

Tabulka 2 obsahuje časovou závislost (v sekundách) jednotlivých data
setů na parametrech \(k\) a \(n\).

\begin{figure}[htp]
\centering
\begin{longtable}[]{@{}llccccc@{}}
\toprule
n & k & Data set 1 & Data set 2 & Data set 3a & Data set 3b & Data set
4 \\
\midrule
\endhead
10000 & 10 & 0.1186 & 0.1286 & 0.1932 & 0.1108 & 0.1240 \\
10000 & 50 & 0.1220 & 0.1296 & 0.1944 & 0.1176 & 0.1252 \\
10000 & 100 & 0.1252 & 0.1328 & 0.1966 & 0.1120 & 0.1264 \\
20000 & 10 & 0.2350 & 0.2494 & 0.4262 & 0.2196 & 0.2448 \\
20000 & 50 & 0.2352 & 0.2536 & 0.4284 & 0.2218 & 0.2462 \\
20000 & 100 & 0.2462 & 0.2560 & 0.4296 & 0.2242 & 0.2492 \\
40000 & 10 & 0.4712 & 0.5074 & 0.9765 & 0.4404 & 0.4942 \\
40000 & 50 & 0.4734 & 0.5130 & 0.9778 & 0.4426 & 0.4954 \\
40000 & 100 & 0.4802 & 0.5152 & 0.9813 & 0.4460 & 0.4998 \\
80000 & 10 & 0.9501 & 1.0326 & 2.5466 & 0.8942 & 1.0118 \\
80000 & 50 & 0.9534 & 1.0345 & 2.5576 & 0.8954 & 1.0138 \\
80000 & 100 & 0.9558 & 1.0404 & 2.5686 & 0.8964 & 1.0140 \\
\bottomrule
\end{longtable}
\captionsetup{labelformat=empty}
\caption{Tabulka 2: Časová závislost na parametrech \(k\) a \(n\) \cite{base}}
\end{figure}

Tabulka 3 obsahuje časovou závislost (v sekundách) jednotlivých data
setů na parametru \(k\). Pro toto měření bylo použito \(n = 10000\). Z
této tabulky lze vidět, že pro malá \(k\) (1-5\% z \(n\)) má algoritmus
složitost nižší než lineární. Pro středně velká \(k\) (10-20\% z \(n\))
má algoritmus lineární složitost a pro velká \(k\) (přes 20\% z \(n\))
má algoritmus kvadratickou složitost.

\begin{figure}[htp]
\centering
\begin{longtable}[]{@{}llllll@{}}
\toprule
k & Data set 1 & Data set 2 & Data set 3a & Data set 3b & Data set 4 \\
\midrule
\endhead
100 & 0.1164 & 0.1196 & 0.1802 & 0.1164 & 0.1264 \\
200 & 0.1318 & 0.1276 & 0.1878 & 0.1254 & 0.1340 \\
400 & 0.1494 & 0.1497 & 0.2132 & 0.1460 & 0.1494 \\
800 & 0.2010 & 0.2000 & 0.2912 & 0.1988 & 0.2022 \\
1600 & 0.4118 & 0.4030 & 0.5790 & 0.4008 & 0.3988 \\
3200 & 1.3250 & 1.2896 & 1.6686 & 1.2950 & 1.2666 \\
6400 & 5.0460 & 4.8026 & 5.2992 & 4.7840 & 5.0344 \\
\bottomrule
\captionsetup{labelformat=empty}
\caption{Tabulka 3: Časová závislost na parametru \(k\) \cite{base}}
\end{longtable}
\end{figure}

Tabulka 4 ukazuje časovou závislost (v sekundách) jednotlivých data setů
na měřítku mřížky(\(\alpha\)). Z tabulky je zřejmé, že nejvýhodnější je
volba \(\alpha=1.5\).

\begin{figure}[htp]
\centering
\begin{longtable}[]{@{}lccccl@{}}
\toprule
\(\alpha\) & Data set 1 & Data set 2 & Data set 3a & Data set 3b & Data
set 4 \\
\midrule
\endhead
0.25 & 1.0760 & 0.9314 & 4.7610 & 0.9568 & 0.8976 \\
0.50 & 0.3290 & 0.2636 & 0.6590 & 0.2668 & 0.2614 \\
0.75 & 0.1980 & 0.1548 & 0.2800 & 0.1528 & 0.1560 \\
1.00 & 0.1176 & 0.1198 & 0.1812 & 0.1164 & 0.1274 \\
1.25 & 0.1010 & 0.1108 & 0.1462 & 0.1032 & \textbf{0.1208} \\
1.50 & 0.0934 & \textbf{0.1086} & 0.1308 & \textbf{0.1000} & 0.1264 \\
1.75 & \textbf{0.0924} & 0.1164 & \textbf{0.1274} & 0.1032 & 0.1464 \\
2.00 & 0.0956 & 0.1274 & 0.1306 & 0.1078 & 0.1834 \\
2.25 & 0.0980 & 0.1482 & 0.1372 & 0.1164 & 0.2218 \\
2.50 & 0.1054 & 0.1758 & 0.1494 & 0.1286 & 0.2450 \\
2.75 & 0.1122 & 0.2032 & 0.1614 & 0.1428 & 0.2704 \\
3.00 & 0.1208 & 0.2252 & 0.1802 & 0.1636 & 0.2932 \\
4.00 & 0.1812 & 0.3142 & 0.2592 & 0.2428 & 0.4294 \\
5.00 & 0.2450 & 0.4350 & 0.3384 & 0.3240 & 0.6174 \\
6.00 & 0.3208 & 0.5811 & 0.4426 & 0.4252 & 0.8480 \\
7.00 & 0.4022 & 0.7558 & 0.5624 & 0.5482 & 1.1062 \\
\bottomrule
\captionsetup{labelformat=empty}
\caption{Tabulka 4: Časová závislost na měřítku mřížky \cite{base}}
\end{longtable}
\end{figure}


\setkeys{Gin}{width=\maxwidth,height=\maxheight,keepaspectratio}
\begin{figure}[htp]
\centering
\includegraphics{./data1.png}
\captionsetup{labelformat=empty}
\caption{Data set 1: uniformně rozdělená data \cite{base}}
\end{figure}

\begin{figure}[htp]
\centering
\includegraphics{./data2.png}
\captionsetup{labelformat=empty}
\caption{Data set 2: díra v datech \cite{base}}
\end{figure}

\begin{figure}[htp]
\centering
\includegraphics{./data3a.png}
\captionsetup{labelformat=empty}
\caption{Data set 3a: díra v datech, hledání probíhá z izolovaného bodu \cite{base}}
\end{figure}

\begin{figure}[htp]
\centering
\includegraphics{./data3b.png}
\captionsetup{labelformat=empty}
\caption{Data set 3b: díra v datech, hledání probíhá z bodu, který není izolovaný \cite{base}}
\end{figure}


\begin{figure}[htp]
\centering
\includegraphics{./data4.png}
\captionsetup{labelformat=empty}
\caption{Data set 4: data jsou rozdělena do dvou množin a vyhledávání probíhá v jedné z nich \cite{base}}
\end{figure}

\hypertarget{zuxe1vux11br}{%
\chapter{Závěr}\label{zuxe1vux11br}}

Představený algoritmus pracuje ve 2-D a je zřejmé, že je výhodné ho využívat pouze pro malé hodnoty \(k\).
Testy ukázali, že časová složitost se příliš nezmění v závislosti na topologii dat. Aby se výsledky testování daly reprodukovat,
je důležité, aby programátor správně naimplementoval uniformní mřížku a správně pracoval s daty. Algoritmus nepotřebuje výrazně 
realokovat paměť. Lineární časová složitost lze očekávat až do 70\% prázdných buněk v mřížce.

\begin{thebibliography}{9}
\bibitem{edelsbrunner}
Edelsbrunner H. Algorithms in combinatorial geometry. New York: Springer-Verlag, 1987

\bibitem{rourke}
O'Rourke J. Computational geometry in C. New York: Cambridge University Press, 1994

\bibitem{preparatafp}
Preparata FP, Shamos MI. Computational geometry: an introduction. New York: Springer-Verlag, 1985
Wesley, Massachusetts, 2nd ed.

\bibitem{base}
\href{https://www.sciencedirect.com/science/article/pii/S001044850000141X}{Piegl, L.A. \& Tiller, W. (2002). Algorithm for finding all k nearest neighbors. Computer-Aided Design 34 (2002) 167-172}

\end{thebibliography}

\end{document}
